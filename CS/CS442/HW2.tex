\documentclass[10pt,a4paper]{article}
\usepackage[utf8]{inputenc}
\usepackage{amsmath}
\usepackage{amsfonts}
\usepackage{amssymb}
\usepackage{graphicx}
\usepackage{kotex}
\usepackage{hyperref}
\usepackage{listings}
\usepackage[super]{nth}
\usepackage[english]{babel}
\graphicspath{{../}}
\hypersetup{
	colorlinks=true,
	linkcolor=blue,
	filecolor=magenta,      
	urlcolor=blue,
}

%% Page Layout %%
\usepackage[left=2.5cm,right=2.5cm,top=3cm,bottom=3cm,a4paper]{geometry}
\setlength{\parskip}{1.5ex}
%
% Homework Details
%   - Title
%   - Due date
%   - Class
%   - Section/Time
%   - Instructor
%   - Author
%

\newcommand{\hwTitle}{Essay \#1}
\newcommand{\DueDate}{October 14, 2019}
\newcommand{\Semester}{Fall 2019}
\newcommand{\Course}{CS442}
\newcommand{\CourseInstructor}{Prof. Sung-Ju Lee}
\newcommand{\AuthorName}{Inyong Koo}
\newcommand{\StudentID}{20160042}

\newcommand{\Paper}{FLUID: Flexible User Interface Distribution for Ubiquitous Multi-device Interaction}

%% Headnote, footnote %%
\usepackage{fancyhdr}
\pagestyle{fancy}

\fancyhf{}
\rhead{\StudentID \, \AuthorName}
\lhead{\Semester}
\chead{\Course \, \hwTitle}
\cfoot{\thepage}

\renewcommand{\headrulewidth}{1pt}
\renewcommand{\footrulewidth}{1pt}

\fancypagestyle{titlepagestyle}
{
	\fancyhf{}
	\rhead{Due date: \DueDate}
	\lhead{\Semester}
	\cfoot{\thepage}
	\renewcommand{\footrulewidth}{1pt}
}

\begin{document}
	\title{\vspace{-7ex}\Course \, Mobile Computing, Networking \& Applications \\ \hwTitle \\ \Large{\textbf{\Paper}}}
	\author{\StudentID \ \AuthorName}
	\date{\hrule}
	%\date{}  % Toggle commenting to test
	
	\maketitle
	
	\thispagestyle{titlepagestyle}
	
	
	The paper presents FLUID, an Android-based platform that offers flexible user interfaces (UIs) distribution for multi-device. In the Introduction, authors emphasize how different FLUID is from other multi-surface applications and screen mirroring applications - what limits the existing approaches share and what FLUID aims to achieve. The demo video of use cases\footnote{Youtube, \href{https://youtu.be/lGO4GwH4enA}{"FLUID: Flexible User Interface Distribution for Ubiquitous Multi-Device Interaction" by KAIST}} helped me comprehend their points. It really appealed to me that FLUID can offer better usability, collaborative use, and privacy protection.
	
	I believe the strength of FLUID comes from its flexibility and applicability. I liked how we can designate individual UI components to migrate. Supporting a wide range of UI types was also impressive. The authors assessed their work in various aspects, and the evaluation results look promising too. Although, I’d like to see a performance test results for the case when more than two devices were paired. I doubt if FLUID will work as fluently as it is in such situations.
	
	That leads to my concern about using RPC as the cross-device communication method. For collaborative works, one may have to distribute UIs to multiple devices. I was thinking Bluetooth, but it's still not enough if remote devices are involved. The authors also mentioned that they expect to use 5G or 802.11ad in later works.
	
	I also have a suggestion regarding the host-guest architecture, once we're going to use the wireless network. I thought it would be much cooler if guest can request for UI migration. On the proposed model, one should receive UI components from the 'main' device, and I believe this somewhat disrupts the user experience. If one can access to multiple hosts by an administrative device or a pre-authorized device by the host, the user can more easily organize the functions they'd like to use. For instance, one can access a television channel controller while using the device as the gaming controller for another tablet device. You can bring UI components from different devices without reaching them.
	
	In order to implement such functionality, I think it would be better to establish a controlling server. Device-to-device host-guest architecture is appropriate when two devices are communicating. If we want to make a real multi-surface computing environment, I believe we need to change to the server-client architecture. So that every device has equal functionality to request and allow UI component migration. It will affect performance negatively, but UI response time showed a 2x to 4x outperformance over screen mirroring approach. I think we can compensate for performance with an advanced wireless network.
	
	I loved this paper. FLUID can be a versatile platform for the approaching generation of IoT. I was fascinated with Samsung's Smart Things, and I do believe that this research opens a new possibility for so many applications. I'm excited to see the related works and hope to find a proceeding approach similar to mine.
		
\end{document}