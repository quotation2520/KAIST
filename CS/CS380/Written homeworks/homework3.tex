\documentclass[10pt,a4paper]{article}
\usepackage[utf8]{inputenc}
\usepackage{amsmath}
\usepackage{amsfonts}
\usepackage{amssymb}
\usepackage{graphicx}
\usepackage{kotex}
\usepackage[super]{nth}
\usepackage[english]{babel}

%% Page Layout %%
\usepackage[left=2.5cm,right=2.5cm,top=3cm,bottom=3cm,a4paper]{geometry}

%
% Homework Details
%   - Title
%   - Due date
%   - Class
%   - Section/Time
%   - Instructor
%   - Author
%

\newcommand{\hwTitle}{Homework\ \#3}
\newcommand{\DueDate}{April 2, 2019}
\newcommand{\Semester}{Spring 2019}
\newcommand{\Course}{CS380}
\newcommand{\CourseInstructor}{Prof. Jinah Park}
\newcommand{\AuthorName}{Inyong Koo}
\newcommand{\StudentID}{20160042}


%% Headnote, footnote %%
\usepackage{fancyhdr}
\pagestyle{fancy}

\fancyhf{}
\rhead{\StudentID \, \AuthorName}
\lhead{\Semester}
\chead{\Course \, \hwTitle}
\cfoot{\thepage}

\renewcommand{\headrulewidth}{1pt}
\renewcommand{\footrulewidth}{1pt}

\fancypagestyle{titlepagestyle}
{
	\fancyhf{}
	\rhead{Due date: \DueDate}
	\lhead{\Semester}
	\cfoot{\thepage}
	\renewcommand{\footrulewidth}{1pt}
}

\begin{document}
	\title{\vspace{-7ex}\Course \, Introduction to Computer Graphics \\ \hwTitle}
	\author{\StudentID \ \AuthorName}
	\date{\hrule}
	%\date{}  % Toggle commenting to test
	
	\maketitle
	
	\thispagestyle{titlepagestyle}
	{\large Describe the following terms with respect to computer graphics.}
	

	\begin{enumerate}
		
		{\large \item Point}
		
			\textbf{Point} is one of the fundamental geometric objects. A point indicates only location.  \\
			
		{\large \item Affine sum}
		
			In an affine space, the addition of two vectors, the multiplication of a vector by a scalar, and the addition of a vector and a point are defined, the addition of two aribitrary points and the multiplication of a point by a scalar are not. 
			However, we can express point P upon $ \overline{RQ} $ using following equation with latter two operations. We call such operation the ‘\textbf{Affine sum}’\\
			$ P = \alpha_1 R + \alpha_2 Q $ where $ \alpha_1 + \alpha_2 = 1 $\\
			
		{\large \item Convex hull}
		
			The \textbf{convex hull} is the set of points formed by the affine sum of n points.
			$ P = \alpha_1 P_1 + \alpha_2 P_2 + \cdots + \alpha_3 P_3 $ where $ \alpha_1 + \alpha_2 + \cdots + \alpha_n = 1 $, under additional restriction $ \alpha_i \geq 0, i = 1, 2, \cdots, n $\\
			Geometrically, the convex hull is the set of points that we form by stretching a tight-fitting surface over the given set of points – shrink-wrapping of the points.\\
			
		{\large \item Barycentric coordinate}
		
			We can express a plane by expressing coordinate of a point in the plane using two nonparallel vectors. Using concept of affine sum, it means a point on a plane can be expressed using three points, not on a single line, of the plane. $ T = \alpha P + \beta Q + \gamma R $ where $ \alpha + \beta + \gamma = 1 $. The representation of a point by ($ \alpha, \beta, \gamma $) is called its \textbf{barycentric coordinates} representation.\\
			
		{\large \item Tessellation}
				
			Since triangular polygons are always flat, either the modeling system is designed to always produce triangles, or the graphics system provides a method to divide an arbitrary polygon into triangular polygons. We call this division into triangles, \textbf{tessellation}.\\
			
		{\large \item CSG (Constructed solid geometry)}
				
			\textbf{Constructive solid geometry (CSG)} is the major exception to triangle mesh approximation approach to provide curved objects. In CSG system, we build objects from a small set of volumetric objects through a set of operations such as union and intersection. It is an excellent approach for modelling, but rendering CSG models is more difficult than is rendering surface-based polygonal models.\\
				
		{\large \item Homogenous coordinates}
				
			Common three-dimensional representation of coordinate can cause confusion between representation of a point and a vector. \textbf{Homogenous coordinates} avoid this difficulty by using a four-dimensional representation for both points and vectors in three dimensions.\\
			Specified by $ (v_1, v_2, v_3, P_0) $, we can represent a point by $ P = \alpha_1 v_1 + \alpha_2 v_2 + \alpha_3 v_3 + P_0 $. P is then represented by $ (\alpha_1, \alpha_2, \alpha_3, 1) $ In the same frame, a vector can be represented by $ w = \delta_1 v_1 + \delta_2 v_2 + \delta_3v_3 $. W is then represented by $ (\delta_1, \delta_2, \delta_3, 0) $.\\
			
		{\large \item Frames (compared to coordinates)}
				
			A \textbf{frame} is determined by an origin and basis vectors. Instead of using only basis vectors as in coordinates, notation of a frame allows us to avoid the difficulties caused by vectors having magnitude and direction but no fixed position. We are also able to represent points and vectors in a manner that will allow us to use matrix representations but that maintains a distinction between the two geometric types.\\
			
		{\large \item Normalized device coordinates}
				
			\textbf{Normalized device coordinates} is a coordinate system occurred during the pipeline, after the clip coordinates are divided by the w component, called perspective division, into three-dimensional representation. The normalized device coordinates later take into account the viewport, and create window coordinates.\\
			
		{\large \item Affine transformations}
				
			 \textbf{Affine transformations} are the transformation takes place in affine space, homogenous coordinates. Transformation is a linear function that has 12 degrees of freedom. Line is conserved in affine transform so if we want to transform a line segment, we need only to transform the endpoints. Rotation, translation, scaling and shearing are the affine transformations.\\
			 
		{\large \item Rigid-body transformation}
				
			 \textbf{Rigid-body transformations} are transformations that does not alter the shape or volume of an object, but only the object’s location and orientation. (Rotation and translation)\\
			 
		{\large \item Shearing}
				
			 Shearing is one of the affine transformations that changes shape of the object, but not its volume. Another affine transformation that is not rigid-body transformation is scaling, which changes volume of the object but not its shape.\\
			 
		{\large \item Uniform variables}
				
			When we send vertex attributes to a shader, these attributes can be different for each vertex in a primitive. We may also want parameters that will remain the same for all vertices in a primitive or equivalently for all the vertices that are displayed when we execute a function. Such variables are called uniform variables.\\
			
		{\large \item Quaternion}
				
			Quaternions are an extension of complex numbers that provide an alternative method for describing and manipulating rotations.\\
			
		{\large \item Zero vector}
			
			\textbf{Zero vector 0} is a vector that satifies $ u + \textbf{0} = u $ for $ \forall u \in V $.\\	
		
		{\large \item Basis}
	
			If a vector space has dimension $ n $, any set of $ n $ linearly independent vectors form a \textbf{basis}. \\
			If $ v_1, v_2, \cdots, v_n $ is a basis for V, any vector $ v $ can be expressed uniquely in terms of the basis vectors as $ v = \beta_1 v_1 + \beta_2 v_2 + \cdots + \beta_n v_n $.
	\end{enumerate}
\end{document}