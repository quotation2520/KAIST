\documentclass[10pt,a4paper]{article}
\usepackage[utf8]{inputenc}
\usepackage{amsmath}
\usepackage{amsfonts}
\usepackage{amssymb}
\usepackage{graphicx}
\usepackage{kotex}
\usepackage{hyperref}
\usepackage{listings}
\usepackage[super]{nth}
\usepackage[english]{babel}
\graphicspath{{../}}

%% Page Layout %%
\usepackage[left=2.5cm,right=2.5cm,top=3cm,bottom=3cm,a4paper]{geometry}
\setlength{\parskip}{1.5ex}
%
% Homework Details
%   - Title
%   - Due date
%   - Class
%   - Section/Time
%   - Instructor
%   - Author
%

\newcommand{\hwTitle}{Assignment \#1}
\newcommand{\DueDate}{November 10, 2019}
\newcommand{\Semester}{Fall 2019}
\newcommand{\Course}{CS489}
\newcommand{\CourseInstructor}{Prof. Shin Yoo}
\newcommand{\AuthorName}{Inyong Koo}
\newcommand{\StudentID}{20160042}

\newcommand{\Question}{Write a 500~1,000 words essay with the theme of “my perspective on ethics”. Feel free to choose a topic that meets the theme: you can pick a specific technical or social issue, or try to describe your broader ethical view.}

%% Headnote, footnote %%
\usepackage{fancyhdr}
\pagestyle{fancy}

\fancyhf{}
\rhead{\StudentID \, \AuthorName}
\lhead{\Semester}
\chead{\Course \, \hwTitle}
\cfoot{\thepage}

\renewcommand{\headrulewidth}{1pt}
\renewcommand{\footrulewidth}{1pt}

\fancypagestyle{titlepagestyle}
{
	\fancyhf{}
	\rhead{Due date: \DueDate}
	\lhead{\Semester}
	\cfoot{\thepage}
	\renewcommand{\footrulewidth}{1pt}
}

\begin{document}
	\title{\vspace{-7ex}\Course \, Computer Ethics \& Social Issues \\ \hwTitle}
	\author{\textbf{Self-evaluation: Am I a Moral Person?} \\\\ \StudentID \ \AuthorName}
	\date{\hrule}
	%\date{}  % Toggle commenting to test
	
	\maketitle
	
	\thispagestyle{titlepagestyle}
	
	\large \textbf {Q. \Question}
	
	\vspace{3ex}
	In the recent lecture (of November \nth{5}), we learned about different perspectives on ethics. I learned many new terms, but I was quite familiar with most of the concepts.
	
	When I was a high school student, I read the first book of \textit{The Republic} by Plato. I always wanted to be a ‘good’ person, but then I noticed that I never asked the relevant questions. What is good? Why should I be good? I noticed that until then, I was merely acting polite; following social conventions and behaving in taught manners. I was shocked by my ignorance, and I started to read diverse philosophy books. After I finished \textit{The Republic}, I moved on to \textit{Nicomachean Ethics} by Aristotle, to \textit{The Prince} by Machiavelli, and the journey went on to relatively recent books, such as \textit{What is Justice} by Michael Sandel.\footnote{It doesn't mean that I understood those books in depth. I was a teenager, and I barely got a glimpse of the arguments} I was also interested in oriental philosophy. I believed if we get rid of some obsolete parts from the Confucianism, we can still find applicable wisdom from the philosophy. I had many debates on the morals of social issues with my friends and occasionally re-established my ethical perspective. 
	
	As I study, I was able to develop my morality. I categorized myself as a deontologist - but rather a flexible one. I followed my conscience, but it did not necessarily agree with the legal obligations. I was a guy who would cross a street in red light without guilt if there is no car around. For me, the law is no more than a social agreement to settle problems when conflict occurs. I put my moral law above those rules.
	
	I was loyal to my conscience. I admired people who walk on the righteous path with good cause. I pursued virtues. I made an ideal - selfless, optimistic, and kind - version of me and tried to be the person. So in my definition, I was a moral person.
	
	However, at the back of my mind, I knew I was not. I knew I was not the saint (or gentleman, 군자 in Korean) I wanted to be. I had selfish, anxious nature in me, and I often hesitated to do the right thing. I confess I sometimes let that weak aspect of me win, and ignored my conscience when I am stressed out. I felt like a hypocrite and blamed myself for not being a more disciplined person.
	
    Anyway, I was confident that I had a concrete ethical standard. I may not be a moral person, but I believed I would know what is right and wrong with certain. But that arrogant misconception shattered in my first year at college.
	
	It was during the school festival. My friends and I were selling hotdogs at the club booth, and the change ran out. Banks were closed, so we needed somewhere else to get coins. I came up with an idea. I suggested that we can exchange coins from the coin-exchanger machine at school karaoke.\footnote{we have coin karaoke facility near the outdoor theater.} Nobody guards the facility, and since we're paying the equivalent amount of money, I thought there's nothing wrong with the idea. But then my best friend laughed out loud and said that I must be joking. He claimed that we shouldn't. Though he didn't even tell me why it's wrong, I felt a rush of embarrassment sweeping over me. The idea, which seemed like a clever one only a moment ago, now felt simply wrong.
	
	I didn't know that I could withdraw my moral judgment that quickly. But it was not the act by surprise. The incident at the school festival was probably nothing. I could say "well, guess I'm wrong then" in a blink because it was not a big deal. But I asked myself, what if we disagree on more pressing issues? What if a person I love claims that what I believe right is wrong?  Or even more, what if he/she wants me to do an immoral action on my standard? I would probably argue. I'll try to persuade him/her so that I could keep my conscience. But what if it's inevitable? If people I love need me to do something against my conscience, what shall I do? Answering the question wasn't difficult. I was ready to get my hands dirty for them. I'd rather live with guilt than let them down.
	
	I realized that people I love mattered to me more than my conscience. It was because that was the basis of my morality. I was never concerned about being a 'good' person. I want to be a person that people I love need. But it doesn't mean that my past conscience, the inquiry in ethics became irrelevant. I built my relationship on my moral ground, so people around me share a similar ethical view with me. They are my external conscience, responsibility.
	
	I am the same person as I was before. But now I am more honest with myself. I learned that I was not seeking justice but acceptance.  I admitted my selfishness and desire to be loved and noticed. I no longer act as if I’m a philanthropist, but devote my love to people I sincerely care about. I now think of my moral laws as the preference, not the obligation. I try to be a good man not because I feel like I should, but because I want to.
	
	I once thought ethics as the study of justice and truth. But perhaps, there is no such thing as absolute goodness. To be honest, I don't care anymore. I just hope I'm good enough for somebody. I study ethics to understand people; to learn their motives and behaviors. I study ethics so that I can respect people around me.  I study ethics to be a better friend, a better companion. So am I a moral person? Well, it's not for me to decide.
		
\end{document}